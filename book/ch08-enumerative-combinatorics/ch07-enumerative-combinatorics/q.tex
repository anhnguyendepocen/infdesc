% !TeX root = ../../book.tex
\subsection*{Finite sets}

\begin{chapex}
Let $n \in \mathbb{N}$ and let $f : [n] \to [n]$ be a function. Prove that $f$ is injective if and only if $f$ is surjective.
\end{chapex}

\begin{chapex}
Prove that $|\mathbb{Z}/n\mathbb{Z}|=n$ for all $n \ge 1$.
\end{chapex}

\begin{chapex}
Let $A$, $B$ and $C$ be sets. Prove that if $A \triangle B$ and $B \triangle C$ are finite, then $A \triangle C$ is finite, where the notation $\triangle$ refers to the \textit{symmetric difference} operation (\Cref{defSymmetricDifference})---that is, $A \triangle B = (A \setminus B) \cup (B \setminus A)$, and so on.
\end{chapex}

\subsection*{Counting}

\begin{chapex}
Let $X$ and $Y$ be finite sets with $|X|=m \in \mathbb{N}$ and $|Y|=n \in \mathbb{N}$. Prove that there are $2^{mn}$ relations from $X$ to $Y$.
\end{chapex}

\begin{chapex}
Let $X$ be a set and let $R$ be a relation on $X$. Prove that $R$ is reflexive if and only if $\Delta_X \subseteq \mathrm{Gr}(R)$, where $\Delta_X$\nindex{DX}{$\Delta_X$}{diagonal subset} is the diagonal subset of $X \times X$ (see \Cref{defDiagonalSubset}). Deduce that if $X$ is finite and $|X|=n \in \mathbb{N}$, then there are $2^{n(n-1)}$ reflexive relations on $X$.
\end{chapex}

\begin{chapex}
Let $X$ be a finite set with $|X|=n \in \mathbb{N}$. Prove that there are $2^{\binom{n}{2}} \cdot 2^n$ symmetric relations on $X$.
\end{chapex}

\begin{chapex}
Let $X$ be a finite set with $|X|=n \in \mathbb{N}$. Prove that there are $3^{\binom{n}{2}} \cdot 2^n$ antisymmetric relations on $X$.
\end{chapex}

\begin{chapex}
Let $X$ be a finite set with $|X| = n \in \mathbb{N}$, let $\sim$ be an equivalence relation on $X$, and suppose that there is some natural number $k$ such that $|[a]_{\sim}| = k$ for all $a \in X$. Prove that $k$ divides $n$, and that $|X/{\sim}| = \dfrac{n}{k}$.
\hintlabel{cqSizeOfQuotient}{%
Prove that $|X/{\sim}| \cdot k = n$ using the multiplication principle: find a two-step procedure for specifying an element of $X$ by first considering its $\sim$-equivalence class.
}
\end{chapex}

\begin{chapex}
Let $n,k \in \mathbb{N}$ with $k \le n$. Prove that the number of functions $f : [n] \to [n]$ that fix exactly $k$ elements of $[n]$ is equal to $\dbinom{n}{k}(n-1)^{n-k}$.
\end{chapex}

\subsection*{Double counting}

\begin{chapex}
Let $a,b,m,n \in \mathbb{N}$. Prove each of the following by double counting.
\begin{multicols}{2}
\begin{enumerate}[(a)]
\item $a(m+n) = am + an$
\item $a^{m+n} = a^m \cdot a^n$
\item $(a^m)^n = a^{mn}$
\item $(ab)^n = a^n \cdot b^n$
\end{enumerate}
\end{multicols}
\end{chapex}

\begin{chapex}
Prove that $\displaystyle \sum_{k=0}^n \binom{n}{k}^2 = \binom{2n}{n}$ for all $n \in \mathbb{N}$
\end{chapex}

\begin{chapex}
Prove that $\displaystyle \sum_{k=m}^n \binom{n}{k} \binom{k}{m} = 2^{n-m} \binom{n}{m}$ for all $m,n \in \mathbb{N}$ with $m \le n$.
\end{chapex}

\begin{chapex}
Prove that $\displaystyle \sum_{j=0}^k \binom{n-j}{k-j} = \binom{n+1}{k}$ for all $n,k \in \mathbb{N}$ with $k \le n$.
\end{chapex}

\begin{chapex}
Prove that $\displaystyle \sum_{k=1}^n \sum_{\ell = 0}^k k \binom{n}{k} \binom{n-k}{\ell} = n \cdot 3^{n-1}$ for all $n \in \mathbb{N}$.
\end{chapex}

\begin{chapex}
Prove that $\displaystyle \binom{n}{r+s+1} = \sum_{k=r+1}^{n-s} \binom{k-1}{r} \binom{n-k}{s}$ for all $n,r,s \in \mathbb{N}$.
\end{chapex}

\begin{chapex}
Let $a_1,a_2,\dots,a_r \in \mathbb{N}$ and let $n = a_1 + a_2 + \cdots + a_r$. Prove that
\[
\binom{n}{a_1,a_2,\dots,a_r} = \prod_{k=0}^{r-1} \binom{n - \sum_{i=1}^k a_i}{a_{k+1}}
\]
\end{chapex}
where $\dbinom{n}{a_1,a_2,\dots,a_r}$ is the number of ordered $r$-tuples $(U_1, U_2, \dots, U_r)$ such that $U_1, U_2, \dots, U_r$ is a partition of $[n]$ and $|U_k| = a_k$ for all $k \in [r]$.

\subsection*{Alternating sums}

\begin{chapex}
Let $X$ be a finite set. Prove that if $|X|$ is odd then there is no parity-swapping involution $X \to X$.
\hintlabel{exNoParitySwappingInvolutionFromOddSizedSetToItself}{%
Find a way to apply \Cref{lemParitySwappingInvolutionInducesBijectionFromEvenToOdd}.
}
\end{chapex}

\begin{chapex}
Find the number of subsets of $[100]$ that do not contain a multiple of $8$.
\end{chapex}

\subsection*{Miscellaneous exercises}

\begin{chapex}
Let $n \in \mathbb{N}$. For each $k \in \mathbb{N}$, find the coefficient of $x^k$ in the polynomial $p(x)$ defined by $p(x) = (1+x+x^2)^n$.
\end{chapex}

\begin{chapex}
Fix $n \in \mathbb{N}$. For each $k \in \mathbb{N}$, find the coefficient of $x^k$ in the polynomial $p(x)$ defined by $p(x) = \sum_{i=0}^n x^{n-i}(1+x)^i$.
\end{chapex}

\subsection*{True--False questions}

\tfquestiontext{cqCombinatoricsTFBegin}{cqCombinatoricsTFEnd}

\begin{chapex} % False
\label{cqCombinatoricsTFBegin}
Every finite set has a unique enumeration.
\end{chapex}

\begin{chapex} % True
The union of a finite number of finite sets is finite.
\end{chapex}

\begin{chapex} % True
Every finite set is a proper subset of another finite set.
\end{chapex}

\begin{chapex} % False
There is a finite set with infinitely many subsets.
\end{chapex}

\begin{chapex} % False
\label{cqCombinatoricsTFEnd}
For every partition $\mathcal{U} = \{ U_0, U_1, \dots, U_n \}$ of a finite set $X$, there is a function $f : X \to X$ that is a parity-swapping involution with respect to $\mathcal{U}$.
\end{chapex}

\subsection*{Always--Sometimes--Never questions}

\asnquestiontext{cqCombinatoricsASNBegin}{cqCombinatoricsASNEnd}

\begin{chapex} % Sometimes
\label{cqCombinatoricsASNBegin}
Let $X$ and $Y$ be finite sets and let $f : X \to Y$ be a function. If $|X| \le |Y|$, then $f$ is injective.
\end{chapex}

\begin{chapex} % Never
Let $X$ be a finite set. Then there is some proper subset $U \subsetneqq X$ such that $|U| = |X|$.
\end{chapex}

\begin{chapex} % Sometimes
Let $X$ be an infinite set. Then $\mathbb{N} \subseteq X$.
\end{chapex}

\begin{chapex} % Always
Let $X$ be a finite set. Then $\mathcal{P}(X)$ is finite.
\end{chapex}

\begin{chapex} % Never
Let $a,b \in \mathbb{R}$ with $a<b$. Then the set $[a,b] \cap \mathbb{Q}$ is finite.
\end{chapex}

\begin{chapex} % Sometimes
Let $X$ and $Y$ be sets with $X$ finite, and let $f : X \to Y$ be an injection. Then $Y$ is finite.
\end{chapex}

\begin{chapex} % Always
\label{cqCombinatoricsASNEnd}
Let $X$ and $Y$ be sets with $X$ finite, and let $f : X \to Y$ be a surjection. Then $Y$ is finite.
\end{chapex}